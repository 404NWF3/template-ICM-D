% MCM/ICM Competition Paper Template
% Based on standard formatting requirements

\documentclass[12pt]{article}

% Packages
\usepackage[utf8]{inputenc}
\usepackage[margin=1in]{geometry}
\usepackage{amsmath,amssymb,amsfonts}
\usepackage{graphicx}
\usepackage{booktabs}
\usepackage{multirow}
\usepackage{algorithm}
\usepackage{algorithmic}
\usepackage{float}
\usepackage{hyperref}
\usepackage{xcolor}

% Formatting
\linespread{1.5} % Double spacing
\setlength{\parindent}{0pt} % No paragraph indent
\setlength{\parskip}{6pt} % Space between paragraphs

% Title
\title{\textbf{Team Number:XXXXX}}
\author{}
\date{}

\begin{document}

\maketitle

% Abstract
\begin{abstract}
\noindent
[Three-paragraph structure: (1) Problem background and model, (2) Method and specific numerical results, (3) Sensitivity analysis and conclusions]
\end{abstract}

% Keywords
\noindent \textbf{Keywords:} optimization, mathematical modeling, [specific techniques used]

% Main content
\section{Introduction}

\subsection{Problem Background}
[Brief context, why problem matters]

\subsection{Restatement of Problem}
[Clear statement of what we will accomplish]

\subsection{Our Approach}
[Overview of modeling approach, mention main techniques]

\section{Assumptions and Justifications}

\begin{enumerate}
\item \textbf{[Assumption 1]:} [Assumption] \textit{[Justification]}
\item \textbf{[Assumption 2]:} [Assumption] \textit{[Justification]}
...
\end{enumerate}

\section{Notation}

\begin{table}[H]
\centering
\begin{tabular}{ll}
\toprule
Symbol & Description \\
\midrule
$x_{ij}$ & [Description] \\
$c_{ij}$ & [Description] \\
...
\bottomrule
\end{tabular}
\caption{Model Notation}
\end{table}

\section{Model Establishment}

\subsection{Model Formulation}

\subsubsection{Decision Variables}
[List and define all decision variables]

\subsubsection{Objective Function}
[Mathematical formulation with explanation]

\subsubsection{Constraints}
[List all constraints with mathematical formulation and rationale]

\subsection{Solution Approach}
[Algorithm choice and justification]

\section{Model Solution}

\subsection{Algorithm Design}
[Detailed algorithm description, possibly with pseudocode]

\subsection{Computational Results}
[Present numerical results with tables and figures]

\section{Sensitivity Analysis}

\subsection{Parameter Selection}
[Which parameters were tested and why]

\subsection{One-Way Sensitivity}
[Single parameter variation results]

\subsection{Two-Way Sensitivity}
[Multiple parameter interaction results]

\section{Model Evaluation}

\subsection{Strengths}
\begin{itemize}
\item [Strength 1]
\item [Strength 2]
...
\end{itemize}

\subsection{Weaknesses}
\begin{itemize}
\item [Weakness 1]
\item [Weakness 2]
...
\end{itemize}

\section{Conclusions}

[Summary of findings, practical implications, future work]

% References
\begin{thebibliography}{9}

\bibitem{ref1}
[Author], \textit{[Title]}, [Journal/Conference], [Year].

\bibitem{ref2}
[Author], \textit{[Title]}, [Publisher], [Year].

...

\end{thebibliography}

% Appendices
\appendix

\section{Appendix: Code}
[Include or reference to complete code]

\section{Appendix: Additional Data}
[Supplementary tables and figures]

\end{document}
